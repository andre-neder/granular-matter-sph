
\documentclass[intern]{cgMA}

\usepackage{csquotes} % Für \enquote
% Dokumentation: https://www.namsu.de/Extra/pakete/Csquotes.html
\usepackage{subcaption} % subfigure/subcaption Umgebungen und \subcaption Befehl
\usepackage{hyperref} % Macht automatisch klickbare Links bei Referenzen etc.
\usepackage{listings} % Für Code-listings o.ä.
\usepackage{graphicx} % Für Graphiken
\usepackage{amsmath,amssymb,amsfonts} % Mathe, Symbole....
\DeclareMathOperator{\Tr}{Tr}
\usepackage{algpseudocode}
\usepackage{algorithm}
\usepackage{float}
\usepackage{gensymb} % Mathe, Symbole....
\usepackage{textcomp} % Symbole....
\usepackage{bm} % \bm{x} Befehl um in Mathe fett zu setzen
\usepackage{cleveref}
\usepackage{blindtext}
\usepackage{titlesec}
\nocite{*}
\usepackage[
    backend=biber,
    style=ieee,
    dashed=false,
    sortlocale=en_US,
    natbib=true,
    url=true, 
    doi=true,
    eprint=false
]{biblatex}
\addbibresource{sources.bib}

\title{Real-time Simulation of Granular Matter using Smoothed Particle Hydrodynamics}
\author{André Neder}
\zweitgutachter{Alexander Maximilian Nilles, M.Sc.}
\zweitgutachterInfo{(Institut für Computervisualistik, AG Computergraphik)}
% \externLogo{7.46cm}{logos/ExternLogoPlaceholder}
% \externName{DIN: NewTechnologies}

\begin{document}
    \maketitle
    \newpage
    \tableofcontents
    \newpage
    \section{Introduction}

    Granular matter consists of discrete, macroscopic particles that interact through contact forces, exhibiting unique behaviors such as flowing, layering or piling. Simulating such behaviors in realistic scenarios is especially challenging, due to the necessity of employing a very high number of discrete elements to achieve visually plausible simulations. Granular materials find diverse applications across geophysics, civil engineering, pharmaceuticals, and other industries, playing crucial roles in phenomena such as landslides, construction, powder mixing, storage and transportation. A powerful technique for simulating such materials is Smoothed Particle Hydrodynamics (SPH). SPH is a meshless Lagrangian method that has been employed a lot in fluid dynamics and can also be used to simulate granular behavior. By combining well-established SPH techniques for simulating granular materials with the latest developments in SPH research, real-time simulations might become achievable. In the pages ahead, I will assess the doability of real-time simulations for granular matter, by combining established methods and recent advancements to achieve a dynamic and interactive implementation.

    \section{Related Work}
    
    \section{Basics}
    \subsection{Granular Matter}
    
    \section{Method}
    \subsection{Smoothed Particle Hydrodynamics}
    Smoothed Particle Hydrodynamics (SPH) is a computational technique widely employed in fluid dynamics and other fields for simulating complex physical phenomena.SPH represents the fluid or material as a collection of discrete particles. Each particle has physical properties such as density, pressure, and velocity. The key innovation lies in smoothing these properties over neighboring particles, allowing for a continuous representation of the material and facilitating the simulation of intricate behaviors, including fluid flow, collisions, and deformations. SPH's versatility extends beyond fluids, making it applicable to problems involving granular materials, astrophysics, and even solid mechanics. 

    
    \subsection{Vulkan}
    
    \section{Implementation}
    
    Density neighbors
    \begin{equation}
        \rho_i = \sum_j V_j \rho_j W(x_i - x_j, h)
        = \sum_j m_j W(x_i - x_j, h)
    \end{equation}
    $ W_{ij} \equiv W(x_i - x_j, h)$\\
    
    Density boundaries
    \begin{equation}
        \rho_i = \sum_b V_b \rho_0 W_{ib}
    \end{equation}
    Unilateral incompressibility 
    \begin{equation}
        \rho_i \leq \rho_0 \; \bot \;  p \geq 0
    \end{equation}
    Velocity relative to the air
    \begin{equation}
        v^2_{i,rel} = |v_a - v_i|^2\ \frac{v_a - v_i}{|v_a - v_i|}
    \end{equation}
    Drag
    \begin{equation}
        F^{drag}_i = \frac{1}{2}\rho_a v^2_{i,rel} C_{D,i}  \omega  A_i
    \end{equation}
    Velocity gradient
    \begin{equation}
        \nabla u_i = \sum_j V_j \nabla W_{ij} u_j^T
    \end{equation}
    Strain
    \begin{equation}
        \varepsilon = \frac{1}{2} (\nabla u_i + \nabla u_i^T)
    \end{equation}
    D
    \begin{equation}
        D_i = \frac{2 m_i^2 \Delta t}{\rho_i^2} \sum_j \frac{1}{\rho_j} \nabla W_{ij}  \nabla W_{ij}^T
    \end{equation}
    $\nabla W_{ij}  \nabla W_{ij}^T \equiv W_{ij} \otimes W_{ij}$\\
    Stress
    \begin{equation}
        s_i = D^{-1} \varepsilon
    \end{equation}
    \begin{equation}
        s_{i, hydrostatic} = \frac{1}{2} \Tr s_i 
    \end{equation}
    \begin{equation}
        s_{i, deviatoric} = s_i - s_{i, hydrostatic} 
    \end{equation}
    $s_i$ refers to $s_{i, deviatoric}$ for further equations

    Drucker-Prager Yield Criterion
    \begin{equation}
        ||s_i|| \leq p_i \sqrt{2} \sin \Theta
    \end{equation}

    Friction Force
    \begin{equation}
        F_i^f = -m_i \sum_{j \neq i} m_j (\frac{s_i}{\rho_i^2} + \frac{s_j}{\rho_j^2})  \nabla W_{ij}
    \end{equation}
    \begin{equation}
        F_i^f = -m_i \sum_{b} V_b \rho_0 (\frac{s_i}{\rho_i^2}) \nabla W_{ib}
    \end{equation}

    Todo: EOS
    Pressure Force
    \begin{equation}
        F_i^p = -m_i \sum_{j \neq i} m_j (\frac{p_i}{\rho_i^2} + \frac{p_j}{\rho_j^2})  \nabla W_{ij}
    \end{equation}
    \begin{equation}
        F_i^p = -m_i \sum_{b} V_b \rho_0 (\frac{p_i}{\rho_i^2}) \nabla W_{ib}
    \end{equation}

    Non pressure forces
    \begin{equation}
        F_i^{adv} = F^g + F_i^f + f_i^{drag}
    \end{equation}

    intermediate velocity
    \begin{equation}
        v_i^{adv} = v_i + \Delta t \frac{F_i^{adv}}{m_i}
    \end{equation}

    intermediate density
    \begin{equation}
        \rho_i^{adv} = \rho_i + \Delta t \sum_j m_j v_{ij}^{adv} \nabla W_{ij}
    \end{equation}

    pressure
    \begin{equation}
        p_i^{l+1} = (1 - \omega) p_i^l + \omega \frac{1}{a_{ii} * \Delta t^2} (\rho_0 - \rho_i^{adv} - \Delta t^2 \psi)
    \end{equation}
    $\omega = 0.5$
    \begin{equation}
        \psi = \sum_j m_j (\sum_j d_{ij}p_j^l - d_{jj}p_j^l - \sum_{k \neq i} d_{jk}p_k^l) \nabla W_{ij}
    \end{equation}
    \begin{equation}
        \sum_{k \neq i} d_{jk}p_k^l = \sum_{k} d_{jk}p_k^l - d_{ji}p_i^l
    \end{equation}
    \begin{equation}
        \rho_i^{l+1} = |p * a_{ii} * \Delta t^2 - (\rho_0 - \rho_i^{adv} - \Delta t^2 \psi)| + \rho_0
    \end{equation}

    \begin{equation}
        a_{ii} = \sum_j m_j (d_{ii} - d_{ji}) \nabla W_{ij}
    \end{equation}
    Integration
    \begin{equation}
        v_i(t + \Delta t) = v_i^{adv} + \Delta t \frac{F_i^p}{m_i}
    \end{equation}
    \begin{equation}
        x_i(t + \Delta t) = x_i + \Delta t v_i(t + \Delta t)
    \end{equation}

    \begin{center} 
        \begin{tabular}{|l|l|l|}
        \hline
        Symbol & Unit & Name \\
        \hline
        $\rho_i$ & $kg/m^3$ & Density of particle $i$ at position $x$ for time $t$ \\
        $\rho_0$ & $kg/m^3$ & Rest density of the simulated material \\
        $V_i$ & $m^3$ & Volume of particle $i$ at position $x$ for time $t$ \\
        \hline
        
        \end{tabular}
    \end{center}
    
    \subsection{Libraries \& NewTechnologies}
    
    \subsection{Datastructures}
    
    \subsection{Algorithm}
    \begin{algorithm}
        \caption{Full Simulation Frame}
        \begin{algorithmic}[1]
        \State find neighbors 
        \State compute $\rho$ and $F^{drag}$ 
        \State compute $s$ 
        \State compute $v$ advection 
        \State compute $\rho$ advection 
        \State $l = 0$
        \While {$l < 2$ $||$ $\rho_{avg} - \rho_0 < \eta $}
            \State compute $d_{ij}p_{j}$
            \State compute $p^{l+1}$
            \State $p(t) = p^{l+1}$
            \State apply yield criterion 
            \State $l = l+1$
        \EndWhile
        \State compute $F^p$ and $F^f$ 
        \State integrate 
        \State advect HR particles 
        \end{algorithmic}
    \end{algorithm}
    
    \subsubsection{Neighborhood Search}
    
    \subsubsection{Density \& Pressure}
    
    \subsubsection{Stress \& Strain}
    
    \subsubsection{Force}
    
    \subsubsection{Rigidbody Interactions}
    
    \subsubsection{Integration}
    
    \subsubsection{Upscaling}
    
    \subsubsection{Visualization}
    
    \subsubsection{User Interface}
    
    \section{Evaluation}
    
    \subsection{Performance}
    
    \section{Conclusion \& Future Work}
    
    \newpage
    \printbibliography
\end{document}