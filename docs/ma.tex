
\documentclass[intern]{cgMA}

\usepackage{csquotes} % Für \enquote
% Dokumentation: https://www.namsu.de/Extra/pakete/Csquotes.html
\usepackage{subcaption} % subfigure/subcaption Umgebungen und \subcaption Befehl
\usepackage{hyperref} % Macht automatisch klickbare Links bei Referenzen etc.
\usepackage{listings} % Für Code-listings o.ä.
\usepackage{graphicx} % Für Graphiken
\usepackage{amsmath,amssymb,amsfonts} % Mathe, Symbole....
\DeclareMathOperator{\Tr}{Tr}
\usepackage{algpseudocode}
\usepackage{algorithm}
\usepackage{float}
\usepackage{gensymb} % Mathe, Symbole....
\usepackage{textcomp} % Symbole....
\usepackage{bm} % \bm{x} Befehl um in Mathe fett zu setzen
\usepackage{cleveref}
\usepackage{blindtext}
\usepackage{titlesec}
\nocite{*}
\usepackage[
    backend=biber,
    style=ieee,
    dashed=false,
    sortlocale=en_US,
    natbib=true,
    url=true, 
    doi=true,
    eprint=false
]{biblatex}
\addbibresource{sources.bib}

\title{Real-time Simulation of Granular Matter using Smoothed Particle Hydrodynamics}
\author{André Neder}
\zweitgutachter{Alexander Maximilian Nilles, M.Sc.}
\zweitgutachterInfo{(Institut für Computervisualistik, AG Computergraphik)}
% \externLogo{7.46cm}{logos/ExternLogoPlaceholder}
% \externName{DIN: NewTechnologies}

\begin{document}
    \maketitle
    \newpage
    \tableofcontents
    \newpage
    \section{Introduction}
    
    \section{Related Work}
    
    \section{Basics}
    \subsection{Granular Matter}
    
    \section{Method}
    \subsection{Smoothed Particle Hydrodynamics}
    Density neighbors
    \begin{equation}
        \rho_i = \sum_j V_j \rho_j W(x_i - x_j, h)
        = \sum_j m_j W(x_i - x_j, h)
    \end{equation}
    $ W_{ij} \equiv W(x_i - x_j, h)$\\
    
    Density boundaries
    \begin{equation}
        \rho_i = \sum_b V_b \rho_0 W_{ib}
    \end{equation}
    Unilateral incompressibility 
    \begin{equation}
        \rho_i \leq \rho_0 \; \bot \;  p \geq 0
    \end{equation}
    Velocity relative to the air
    \begin{equation}
        v^2_{i,rel} = |v_a - v_i|^2\ \frac{v_a - v_i}{|v_a - v_i|}
    \end{equation}
    Drag
    \begin{equation}
        F^{drag}_i = \frac{1}{2}\rho_a v^2_{i,rel} C_{D,i}  \omega  A_i
    \end{equation}
    Velocity gradient
    \begin{equation}
        \nabla u_i = \sum_j V_j \nabla W_{ij} u_j^T
    \end{equation}
    Strain
    \begin{equation}
        \varepsilon = \frac{1}{2} (\nabla u_i + \nabla u_i^T)
    \end{equation}
    D
    \begin{equation}
        D_i = \frac{2 m_i^2 \Delta t}{\rho_i^2} \sum_j \frac{1}{\rho_j} \nabla W_{ij}  \nabla W_{ij}^T
    \end{equation}
    $\nabla W_{ij}  \nabla W_{ij}^T \equiv W_{ij} \otimes W_{ij}$\\
    Stress
    \begin{equation}
        s_i = D^{-1} \varepsilon
    \end{equation}
    \begin{equation}
        s_{i, hydrostatic} = \frac{1}{2} \Tr s_i 
    \end{equation}
    \begin{equation}
        s_{i, deviatoric} = s_i - s_{i, hydrostatic} 
    \end{equation}
    $s_i$ refers to $s_{i, deviatoric}$ for further equations

    Drucker-Prager Yield Criterion
    \begin{equation}
        ||s_i|| \leq p_i \sqrt{2} \sin \Theta
    \end{equation}

    Friction Force
    \begin{equation}
        F_i^f = -m_i \sum_{j \neq i} m_j (\frac{s_i}{\rho_i^2} + \frac{s_j}{\rho_j^2})  \nabla W_{ij}
    \end{equation}
    \begin{equation}
        F_i^f = -m_i \sum_{b} V_b \rho_0 (\frac{s_i}{\rho_i^2}) \nabla W_{ib}
    \end{equation}

    Todo: EOS
    Pressure Force
    \begin{equation}
        F_i^p = -m_i \sum_{j \neq i} m_j (\frac{p_i}{\rho_i^2} + \frac{p_j}{\rho_j^2})  \nabla W_{ij}
    \end{equation}
    \begin{equation}
        F_i^p = -m_i \sum_{b} V_b \rho_0 (\frac{p_i}{\rho_i^2}) \nabla W_{ib}
    \end{equation}

    Non pressure forces
    \begin{equation}
        F_i^{adv} = F^g + F_i^f + f_i^{drag}
    \end{equation}

    intermediate velocity
    \begin{equation}
        v_i^{adv} = v_i + \Delta t \frac{F_i^{adv}}{m_i}
    \end{equation}

    intermediate density
    \begin{equation}
        \rho_i^{adv} = \rho_i + \Delta t \sum_j m_j v_{ij}^{adv} \nabla W_{ij}
    \end{equation}

    pressure
    \begin{equation}
        p_i^{l+1} = (1 - \omega) p_i^l + \omega \frac{1}{a_{ii} * \Delta t^2} (\rho_0 - \rho_i^{adv} - \Delta t^2 \psi)
    \end{equation}
    $\omega = 0.5$
    \begin{equation}
        \psi = \sum_j m_j (\sum_j d_{ij}p_j^l - d_{jj}p_j^l - \sum_{k \neq i} d_{jk}p_k^l) \nabla W_{ij}
    \end{equation}
    \begin{equation}
        \sum_{k \neq i} d_{jk}p_k^l = \sum_{k} d_{jk}p_k^l - d_{ji}p_i^l
    \end{equation}
    \begin{equation}
        \rho_i^{l+1} = |p * a_{ii} * \Delta t^2 - (\rho_0 - \rho_i^{adv} - \Delta t^2 \psi)| + \rho_0
    \end{equation}

    \begin{equation}
        a_{ii} = \sum_j m_j (d_{ii} - d_{ji}) \nabla W_{ij}
    \end{equation}
    Integration
    \begin{equation}
        v_i(t + \Delta t) = v_i^{adv} + \Delta t \frac{F_i^p}{m_i}
    \end{equation}
    \begin{equation}
        x_i(t + \Delta t) = x_i + \Delta t v_i(t + \Delta t)
    \end{equation}
    
    \subsection{Vulkan}
    
    \section{Implementation}
    
    \subsection{Libraries \& NewTechnologies}
    
    \subsection{Datastructures}
    
    \subsection{Algorithm}
    \begin{algorithm}
        \caption{Full Simulation Frame}
        \begin{algorithmic}[1]
        \State find neighbors 
        \State compute $\rho$
        \State compute drag 
        \State compute $s$ 
        \State compute $v$ advection 
        \State compute $\rho$ advection 
        \State $l = 0$
        \While {$l < 2$ $||$ $\rho_{avg} - \rho_0 < \eta $}
            \State compute $d_{ij}p_{j}$
            \State compute $p^{l+1}$
            \State $p(t) = p^{l+1}$
            \State $l = l+1$
        \EndWhile
        \State compute $F^p$ 
        \State integrate 
        \State advect HR particles 
        \end{algorithmic}
    \end{algorithm}
    
    \subsubsection{Neighborhood Search}
    
    \subsubsection{Density \& Pressure}
    
    \subsubsection{Stress \& Strain}
    
    \subsubsection{Force}
    
    \subsubsection{Rigidbody Interactions}
    
    \subsubsection{Integration}
    
    \subsubsection{Upscaling}
    
    \subsubsection{Visualization}
    
    \subsubsection{User Interface}
    
    \section{Evaluation}
    
    \subsection{Performance}
    
    \section{Conclusion \& Future Work}
    
    \newpage
    \printbibliography
\end{document}